\documentclass{article}

\usepackage{color}
\usepackage{xcolor}
\usepackage{listings}
\usepackage{polski}
\usepackage[utf8]{inputenc}
\usepackage{caption}
\DeclareCaptionFont{white}{\color{white}}
\DeclareCaptionFormat{listing}{\colorbox{orange}{\parbox{\textwidth}{Przykład: #3 }}}
\captionsetup[lstlisting]{format=listing,labelfont=white,textfont=white}
\title{Notatki do Metod Programowania}
\author{Michał Bronikowski}
\usepackage{url}

\begin{document}
\maketitle
\newpage
\tableofcontents
\newpage
\section{Wstęp do Prologa}
\subsection{Kompilowanie}
Kompilator SWI-PROLOG.\\
W terminalu komenda:

\begin{itemize}
\item swipl lub prolog
\end{itemize}
?- //Działa\\
Kompilacja:\\
Pliki zapisuję z rozszerzeniem .pl. W otwartej "maszynie" prologa wpisuję:\\
?- [p1].  //p1 - nazwa pliku
\subsection{Fakty,Zmienne,Koniunkcje}
lubi(jan,maria).\\
\begin{itemize}
\item Fakt musi się kończyć kropką
\item lubi(jan,\_) - chodzi nam tylko o odpowiedź nie true oe false
\item  po uzyskaniu odpowiedźi jak klikniemy ';' to uzyskamy kolejną o ile istnieje kończymy Enterem
\item koniunkcje oznaczamy ',' 
\begin{lstlisting}[caption=Plik p1.pl]
 lubi(jan,reksio).
 lubi(reksio,bartek).
 lubi(jan,szklanka).
 lubi(jan,beata).
\end{lstlisting}
\begin{lstlisting}[caption=Działanie]
 ?- [p1]. 
 ?- lubi(jan ,beata) , lubi(reksio,bartek).
 true
\end{lstlisting}
\end{itemize}
\subsection{Reguły}
W prologu reguł używa się do zapisania, że fakt zależy od grupy innych faktów.(W języku polskim do stosowania reguł używa się "jeśli").
\begin{lstlisting}[caption = Kot lubi każdego kto lubi mleko]
 czyli:
 Kot lubi wszystko,jesli to lubi mleko,
 Kot lubi X,jesli X lubi mleko.
 ~~~~~~~~~~~~~~~~~~~~~~~~~~~~~~~~~
 lubi(kot,X) :- lubi(X,mleko).

\end{lstlisting}
\subsection{Struktury}
Struktury w Prologu zapisujemy podając funktor oraz jego składniki. Nazwa funktor odpowiada typom z tradycyjnych języków programowania.Składniki ujęte są w nawiasach okrągłych i oddzielone od siebie przecinkami. Funktor umieszcza się przed nawiasem otwierającym.
\begin{lstlisting}[caption = Strukturę można rozbudowywać]
posiada(jan,rower(wigry(niebieski),1991).

\end{lstlisting}


Jan posiada rower marki wigry koloru niebieskiego z 1991 roku
\subsection{Operatory}
Operatory nie powodują wykonanai jakichkolwiek obliczeń 3+4 to nie 7 to term +(3,4).
\begin{itemize}
\item X \texttt{=:=} X  - X i Y są tę samą liczbą
\item X \texttt{=\url{\=}} X  - X i Y są różnymi liczbami
\item X \texttt{<} Y - X jest mniejsze od Y
\item X \texttt{>} Y  - X jest większe od Y
\item X \texttt{=<} Y  - X jest mniejsze równe Y
\item X \texttt{>=} Y  - X jest większe równe Y
\end{itemize}
Operator \textbf{is} operator infiksowy jego prawy argument jest termem, który ma być zinterpretowany jako wyrażenie arytmetyczne. Aby uzgodnic wyrażenie Prolog najpierw oblicza wyrażenie arytmetyczne, a wynik dopasowuje do lewego argumentu
\begin{lstlisting}[caption=Operator "is"]
 ?- X is 2+5.
 X = 5

\end{lstlisting}
Po prawej stronie operatora is można używac takich wyrażeń jak:
\begin{itemize}
\item +
\item -
\item *
\item / - iloraz
\item // całkowity iloraz
\item mod reszta z dzielenia
\end{itemize}
\begin{lstlisting}[caption=Dodawanie]
dodaj(X,Y,Z) :-Z is X + Y.
////
?- dodaj(2,3,A).
A=5.
\end{lstlisting}
\subsection{Listy}
Lista to struktura danych, która jest ciągiem uporządkowanych elementów(dowolne termy). Głowa listy to pierwszy element, ogon to całą reszta. Zapis  \textbf{\url{[X | Y]}} utożsamaia X z głową listy a Y z ogonem.
\begin{lstlisting}[caption=Utożsamianie głowy i ogona]
a([1,2,3]).
~~~~~~~~~~~~~~
?- p([X | Y]).
X=1  Y=[2,3].
\end{lstlisting}
\subsection{Przeszukiwanie rekurencyjne}
Sprawdzmy czy jakis obiekt X jest członkiem listy Y.
Sprawdzamy dwa warunki:
\begin{itemize}
\item X jest elementem Y, gdy jest jego głową
\item X jest elementem Y, gdy X nalezy do ogana tej listy
\end{itemize}
\begin{lstlisting}[caption=Rekurencja]
member(X,[X,_]). /*X nalezy jesli jest glowa*/
member(X,[_ | Y]) :- member(X,Y)./*X nalezy jesli jest w ogonie*/
\end{lstlisting}
Drugi warunek za każdym razem wywołuje krótszą listę. Więc albo uruchomiona zostanie pierwsza klauzula member albo
jako drugi argument zostanie przekazana lista długości 0. Sa to warunki graniczne kończące wywoływanie celów member.
\end{document}