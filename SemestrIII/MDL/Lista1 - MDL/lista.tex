\documentclass[fleqn]{article}
\usepackage[left=1in, right=1in, top=1in, bottom=1in]{geometry}
\usepackage{mathexam}
\usepackage{amsmath}
\usepackage{graphicx} 
\usepackage{latexsym}
\usepackage{amsfonts}
\usepackage{polski}
\usepackage[utf8]{inputenc}
\usepackage{listings}
\usepackage{color}
\definecolor{codegreen}{rgb}{0,0.6,0}
\definecolor{codegray}{rgb}{0.5,0.5,0.5}
\definecolor{codepurple}{rgb}{0.58,0,0.82}
\definecolor{backcolour}{rgb}{0.95,0.95,0.92}
\lstdefinestyle{mystyle}{
    backgroundcolor=\color{backcolour},   
    commentstyle=\color{codegreen},
    keywordstyle=\color{magenta},
    numberstyle=\tiny\color{codegray},
    stringstyle=\color{codepurple},
    basicstyle=\footnotesize,
    breakatwhitespace=false,         
    breaklines=true,                 
    captionpos=b,                    
    keepspaces=true,                 
    numbers=left,                    
    numbersep=5pt,                  
    showspaces=false,                
    showstringspaces=false,
    showtabs=false,                  
    tabsize=2
}
 
\lstset{style=mystyle}

\ExamClass{MDL}
\ExamName{Lista I}
\ExamHead{\today}
\author{Michał Bronikowski}
\let\ds\displaystyle

\begin{document}
\ExamInstrBox{
\begin{center}
Michał Bronikowski \\
Zadania numer: 1,2,7,8
\end{center}
}
\hfill
\\
\\
\\
\\
\\
\\
\[\huge \bf Zadanie \ 1\] 
Udowodnij, że liczba funkcji różnowartościowych z $n$-elementowego zbiory $A$ w $m$ elementowy zbiór B wynosi $\frac{m!}{(m-n)!} $. \\
\\Rozwiązanie: \\ \\
Tworząc funkcje różnowartościowe z $A$ w $B$ na początku wybieram pierwszą wartość f-cji na $m$ sposobów drugą na $m-1$ trzecią na $m-2$ itd. W ogólności: \\ 
\begin{center}
$
m(m-1)(m-2)...(m-n+1) = \frac{m!}{(m-n)!} 
$
\end{center}

 \[\huge \bf Zadanie \ 2\] 
Czy wśród liczb $1,2,...,10^{10}$ zapisanych w systemie dziesiętnym jest więcej tych zawierających cyfrę 9, czy tych, które jej nie zawierają?
\newline \\
Rozwiązanie:
\newline \\
Wszystkich liczb w tym przedziale jest $10^{10}$, liczb nie zawierających $9$ jest $9^{10}$. Ilość liczb zawierających $9$ jest równa: $10^{10} - 9^{10} $. Można ułożyć funkcję:$\frac{10^{10} - 9^{10}}{10^{10}}$
która będzie określała zależność pomiędzy ilością liczb zawierających $9$, a tymi, które jej nie zawierają w tym przedziale. Można to uprościć do postaci $ 1 - (0.9)^{n} $ widać, że jest to funkcja rosnąca, ponieważ $(0.9)^{n}$ maleje dla n = 10 jest mniejsze od $0.5$ więc jest więcej tych zawierających $9$.
 \newpage
 \[\huge \bf Zadanie \ 7\] 
Ile jest możliwych rejestracji samochodowych złożonych z 3 liter, po których następują 4 cyfry?
\newline \\
Rozwiązanie:
\newline \\
Załóżmy, że alfabet ma $21$ znaków (bez polskich znaków), cyfr jest $10$.
Mamy 3 miejsca na litery więc możemy je tam wstawić na $21^{3}$ sposobów. Na cyfry mamy 4 miejsca umieszczamy je tam na $10^{4}$ sposobów.
Ostatecznie mamy $21^{3} \cdot 10^{4} $ możliwych tablic rejestracyjnych.
\[\huge \bf Zadanie \ 8\]
Pokaż, że dla dowolnej liczby rzeczywiatej $x$ i dowolnej liczby całkowitej $n$ zachodzi $ \lceil x + n \rceil = \lceil x \rceil + n$.
\newline \\
Rozwiązanie :
\newline \\
$ \lceil x + n \rceil = min \{ k \in \mathbb{Z} : k \geq x + n \}$
\\Ale wiemy, że $ n \in \mathbb{Z} $ więc, aby $k$ było $\qeq$ niż $x+n$ wystarczy zastosować funkcję $\lceil\rceil$ na $x$ i uzyskamy interesującą nas zależność.
\end{document}

