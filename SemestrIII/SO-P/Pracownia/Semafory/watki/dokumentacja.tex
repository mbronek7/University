\documentclass[10pt,a4paper]{article}
\usepackage{polski}
\usepackage[utf8]{inputenc}
\usepackage{amsmath}
\usepackage{amsfonts}
\usepackage{amssymb}

\usepackage{inconsolata}
\renewcommand*\familydefault{\ttdefault} %% Only if the base font of the document is to be typewriter style
\usepackage[T1]{fontenc}


\usepackage{secdot}  % kropka po numerach sekcji
\usepackage{graphicx} % Required for the inclusion of images
\usepackage{geometry}
 \geometry{
 a4paper,
 total={170mm,257mm},
 left=25mm,
 right=25mm,
 top=20mm,
 }

% Kropki w spisie tresci
\usepackage[dotinlabels]{titletoc}
\titlecontents{chapter}
  [1.5em]
  {\addvspace{\baselineskip}}
  {\contentslabel{1.5em}\hspace*{0em}}
  {}
  {\titlerule*[1pc]{.}\contentspage}

%kod
\usepackage{listings}
\usepackage{color}
 
\definecolor{codegreen}{rgb}{0,0.6,0}
\definecolor{codegray}{rgb}{0.5,0.5,0.5}
\definecolor{codepurple}{rgb}{0.58,0,0.82}
\definecolor{backcolour}{rgb}{0.95,0.95,0.92}
 
\lstdefinestyle{mystyle}{
    backgroundcolor=\color{backcolour},   
    commentstyle=\color{codegreen},
    keywordstyle=\color{magenta},
    numberstyle=\tiny\color{codegray},
    stringstyle=\color{codepurple},
    basicstyle=\footnotesize,
    breakatwhitespace=false,         
    breaklines=true,                 
    captionpos=b,                    
    keepspaces=true,                 
    numbers=left,                    
    numbersep=5pt,                  
    showspaces=false,                
    showstringspaces=false,
    showtabs=false,                  
    tabsize=2,
    basicstyle=\normal    % rozmiar
}
 
\lstset{style=mystyle}
\renewcommand{\lstlistingname}{} %nazwa listingu
%
% instrukcjua jak z pliku to : \lstinputlisting[language=Ruby]{a.rb}
%
% jak nnormalnie to: \begin{lstlisting}[language=Ruby, caption=Ruby example]
%  rozmiary duzy kod: \footnotesize  albo \small
% zdjecia:
% \begin{center}
%  \includegraphics[width=\textwidth]{nazwazdjeciabezrozszerzenia}
% \end{center}
%
% Kolory 
% \colorlet{Fuksjowy}{RubineRed!70!} i potem \textcolor{Fuksjowy}{Lorem ipsum...}
%
%
\usepackage[dvipsnames]{xcolor}
\definecolor{Mycolor2}{HTML}{FFFFED}
\pagecolor{Mycolor2}
\color{RoyalBlue}
%fuksjowy
\colorlet{Fuksjowy}{RubineRed!70!}

\title{Michał Bronikowski\protect\\ \hfill \protect\\ \protect\\ Systemy Operacyjne 2017 \protect\\ problem złodzieja jabłek \protect\\  {\small{wersja oparta na wątkach} }} % Title


\begin{document}
\maketitle  % Insert the title, author and date
\thispagestyle{empty}
\vfill
%© \scriptsize{Michał Bronikowski 2017}
\newpage
\tableofcontents
\newpage
\normalfont
\large
\section{Opis zadania}
\subsection{Problem złodzieja jabłek}
Problem został przeze mnie wymyślony na potrzeby zadania. Wzorowałem się na klasycznym problemie synchronizacji tj. problemie producenta i konsumenta.
Powiedzmy, że jesteśmy złodziejem jabłek i w naszej miejscowości są dwa sady A i B, w których w dość szybkim tempie przybywa jabłek. Kradniemy jabłka z
sadu A lub z sadu B. W międzyczasie w sadach, dzięki działaniu właścicieli i natury jabłek przybywa. Problem polega na takim zsynchronizowaniu procesów
sadów A i B oraz procesu złodzieja, żeby złodziej nie ukradł wszystkich jabłek z sadów, ponieważ to zwróciłoby uwagę właściciela i nasz złodziej mógłby mieć
problemy.
\subsection{Rozwiązanie}
W rozwiązaniu korzystam z trzech semaforów:
\begin{itemize}
\item[•] skradziono1
\item[•] skradziono2
\item[•] uroslo
\end{itemize}
Złodziej czeka aż w sadach przybędzie jabłek co będzie skutkowało opuszczeniem semafara \textcolor{Fuksjowy}{uroslo}. Po czym przystępują do kradzieży jabłek z losowo wybranego przez siebie sadu. Po kradzieży zwalnia się semafor \textcolor{Fuksjowy}{skradziono1} albo \textcolor{Fuksjowy}{skradziono1}, a zamyka \textcolor{Fuksjowy}{uroslo}. Po zwolnieniu semaforów \textcolor{Fuksjowy}{skradziono1} i \textcolor{Fuksjowy}{skradziono2} w sadach ponownie rusza, produkcja jabłek.W celu zabezpieczenia dostępu do sekcjii krytycznych programu użyłem Mutex'ów.
\section{Implementacja}
Program należy uruchamiać na komputerze pod kontrolą systemu operacyjnego Linux. Program składa się z jednego pliku źródłowego sem.c. Załączony został również Makefile pomagający skompilować program.
\begin{lstlisting}[language=bash,caption=Skompilowanie programu poleceniem make]
  $ make
\end{lstlisting}
Program uruchamiamy poleceniem:
\begin{lstlisting}[language=bash,caption=Uruchomienie programu]
  $ ./sem
\end{lstlisting}
\subsection{Struktura programu}
Program składa się z jednego pliku źródłowego sem.c.
W programie możemy podzielić na następujące części:\newpage
\begin{lstlisting}[language=C,caption=Funkcja główna,basicstyle=\small]
  int main()
{
    srand(time(0));              
    pthread_attr_t attr;
    pthread_attr_init(&attr);
    pthread_attr_setdetachstate(&attr, PTHREAD_CREATE_JOINABLE);

    pthread_mutex_init(&mutex, NULL);
    sem_init(&skradziono1, 0, 5);
    sem_init(&skradziono2, 0, 5);
    sem_init(&uroslo, 0, 0);
     
    pthread_t threads[3];
    pthread_create(&threads[0], &attr, sadA, NULL);
    pthread_create(&threads[2], &attr, sadB, NULL);
    pthread_create(&threads[1], &attr, zlodziej, NULL);
    pthread_create(&threads[2], &attr, sadB, NULL);
    pthread_join(threads[0], NULL);
    pthread_join(threads[1], NULL);
    pthread_join(threads[2], NULL);

    pthread_attr_destroy(&attr);
    pthread_mutex_destroy(&mutex);
    sem_destroy(&skradziono1);
    sem_destroy(&uroslo);
    sem_destroy(&skradziono2);

    pthread_exit(NULL);
    return 0;
}
\end{lstlisting}

\begin{lstlisting}[language=C,caption=Funkcja odpowiadająca za złodzieja,basicstyle=\small]
void *zlodziej()
{   
    while(1)
    {
      sem_wait(&uroslo);
      pthread_mutex_lock(&mutex);
      int sad = (rand()%2)+1; 
      if(sad == 1)
      {
          if(jablko_z_A < 1)
          {
              printf("Nie ukradne z sadu A - nie maja juz jablek\n");
              return 0;
          } 
          else
          {
          --jablko_z_A;
          zA++;
          printf("Kradne jablko z sadu A mam juz %d jablek z sadu A i %d jablek z sadu B \n ",zA,zB);
          }
          sem_post(&skradziono2);
      }
      else
      {   if(jablko_z_B < 1)
          {
              printf("Nie ukradne z sadu B - nie maja juz jablek\n");
              return 0;
          }
          else
          {
          --jablko_z_B;
          zB++;
          printf("Kradne jablko z sadu B mam juz %d jablek z sadu A i %d jablek z sadu B \n ",zA,zB);
          }
          sem_post(&skradziono1);
      }
          pthread_mutex_unlock(&mutex);
    }
}
\end{lstlisting}

\begin{lstlisting}[language=C,caption=Funkcja odpowiadająca za sad "B",basicstyle=\small]
void *sadB() {
    while (1)
    {
        sem_wait(&skradziono1);
        pthread_mutex_lock(&mutex);
        printf("Jestem z sadu  B i mam %d jabłek\n", ++jablko_z_B);
        pthread_mutex_unlock(&mutex);
        sem_post(&uroslo);
    }
}
\end{lstlisting}

\begin{lstlisting}[language=C,caption=Funkcja odpowiadająca za sad "A",basicstyle=\small]
void *sadA() {
    while (1)
    {
        sem_wait(&skradziono2);
        pthread_mutex_lock(&mutex);
        printf("Jestem z sadu A i mam %d jabłek\n", ++jablko_z_A);
        pthread_mutex_unlock(&mutex);
        sem_post(&uroslo);
    }
}
\end{lstlisting}

\subsection{Jak działa program i czemu działa?}
Na początku inicjalizuję semafory o nazwach: skradziono,uroslo,skradziono2. Inicjalizuję również mutex dający dostęp do zasobów tylko dla jednego wątku.
Następnie tworzę trzy wątki: sadA,sadB,zlodziej. Procesy odpowiadające
za „produkcję” jabłek w sadach są zwalnianie przez semafory skradziono i
skradziono2. Każdy z procesów sadów informuje o tym, że zwiększyła się ilość
dostępnych jabłek. Proces złodzieja losowo wybiera, z którego sadu chce ukraść
jabłka, jeżeli w danym sadzie nie ma już jabłek oznacza to, że synchronizacja nie
działa poprawnie i wtedy program kończy swoje działanie. Każdorazowo po 
kradzieży złodziej zwalnia odpowiedni semafor przypisany do sadu A lub sadu B.
Na starcie w każdym sadzie jest dziesięć jabłek oraz wartości odpowiadających
im semaforom ustawione są na pięć, a złodzieja na zero. Taki zabieg pozwala
nam uniknąć sytuacji, w której złodziej będzie chciał ukraść jabłka z sadu, w
którym tych jabłek nie ma.
\end{document}