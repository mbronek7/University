\documentclass[10pt,a4paper]{article}
\usepackage{polski}
\usepackage[utf8]{inputenc}
\usepackage{amsmath}
\usepackage{amsfonts}
\usepackage{amssymb}
\usepackage{courier}
%\renewcommand*\familydefault{\ttdefault} %% Only if the base font of the document is to be typewriter style
\usepackage[T1]{fontenc}
\fontsize{12}{15}
\usepackage[dvipsnames]{xcolor}
\definecolor{Mycolor2}{HTML}{FFFFED}
%\pagecolor{Mycolor2}
\usepackage{tocloft}
\renewcommand{\cftsecleader}{\cftdotfill{\cftdotsep}}
%\color{RoyalBlue}
\renewcommand{\cftsecaftersnum}{.}
%fuksjowy
\colorlet{Fuksjowy}{RubineRed!70!}
\usepackage{secdot}
\usepackage{graphicx} % Required for the inclusion of images
\title{Michał Bronikowski\protect \\ \hfill \\ ortograf.pl \\Przykład ciekawej i pożytecznej aplikacji WWW } % Title
\begin{document}
\maketitle  % Insert the title, author and date
\thispagestyle{empty}
\vfill
%© \scriptsize{Copyright by Burgundowi (Michał Bronikowski i Radosław Madzia)}
\newpage
\tableofcontents
\newpage
\large
\section{O stronie}
Ortograf jest aplikacją internetową, która jest systemem sprawdzającym podany tekst pod względem poprawności ortograficznej, stylistycznej, gramatycznej i typograficznej. Jak deklarują twórcy ortografia i interpunkcja sprawdzane są pod kontem 1000 reguł. Oprócz języka polskiego serwis wspiera 28 innych języków.
\section{O interfejsie serwisu}
Interfejs aplikacji jest prosty, nie mamy do czynienia z nadmiarem i niepotrzebnym przepychem. Strona główna jest przejrzysta i zawiera same potrzebne informacje i dwa pola z reklamami, do których położenia można mieć pewne zastrzeżenia, ale cieszmy się, że autorzy serwisu woleli postawić na dochód z reklam niż z płatnej subskrypcji.
\section{Jak trafiłem na ten serwis}
Zasadniczo nigdy nie szukałem serwisu o podobnym charakterze, jest to pierwsza taka strona, z której miałem okazję korzystać. Powiedzmy, że ortografia i interpunkcja to nie są moje najmocniejsze strony do poszukiwania aplikacji o podobnym zastosowaniu, natchnął mnie prowadzący zajęcia no mojej uczelni Zdzisław Płoski. Cieszę się, że skorzystałem z jego rady używanie ortografa, nie opóźnia mojej pracy, ale znacząco zwiększa jej jakość. Żałuję, że wcześniej nie wpadł mi do głowy pomysł poszukania serwisu o podobnej funkcjonalności. Polecam wszystkich wypróbowanie tego serwisu.
\section{Czy mam jakieś uwagi?}
Osobiście nie mam żadnych uwag co do ortografa. W moim przypadku jego funkcjonalność jest wystarczająca, ale zdaję sobie sprawę, że są uzytkownicy, którym ortograf nie wystarczy. Osobiście jestem przekonany, że ortograf stanie się dość często używaną przeze mnie aplikacją, zresztą już się stał.
\end{document}