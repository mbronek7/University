\documentclass[fleqn]{article}
\usepackage[left=1in, right=1in, top=1in, bottom=1in]{geometry}
\usepackage{mathexam}
\usepackage{amsmath}
\usepackage{polski}
\usepackage[utf8]{inputenc}
\usepackage{listings}
\usepackage{color}
\definecolor{codegreen}{rgb}{0,0.6,0}
\definecolor{codegray}{rgb}{0.5,0.5,0.5}
\definecolor{codepurple}{rgb}{0.58,0,0.82}
\definecolor{backcolour}{rgb}{0.95,0.95,0.92}
\lstdefinestyle{mystyle}{
    backgroundcolor=\color{backcolour},   
    commentstyle=\color{codegreen},
    keywordstyle=\color{magenta},
    numberstyle=\tiny\color{codegray},
    stringstyle=\color{codepurple},
    basicstyle=\footnotesize,
    breakatwhitespace=false,         
    breaklines=true,                 
    captionpos=b,                    
    keepspaces=true,                 
    numbers=left,                    
    numbersep=5pt,                  
    showspaces=false,                
    showstringspaces=false,
    showtabs=false,                  
    tabsize=2
}
 
\lstset{style=mystyle}

\ExamClass{ANL}
\ExamName{Lista I}
\ExamHead{\today}
\author{Michał Bronikowski}
\let\ds\displaystyle

\begin{document}
\ExamInstrBox{
\begin{center}
Michał Bronikowski \\
Deklaruję zadania numer: 1,2,3,4,5,6
\end{center}
}
\hfill
\\
\begin{enumerate}
   \item Wiadomo, że rozwiązaniem równania kwadratowego \[ax^{2}+bx+c=0 (a\neq 0)\] są liczby:
   
      \begin{enumerate}
	 \item $\frac{-b+\sqrt{b^{2}-4ac}}{2a}$
	 \item $\frac{-b-\sqrt{b^{2}-4ac}}{2a}$
      \end{enumerate}
      Pokaż na kilku przykładach, że bezpośrednie stosowanie tych wzorów może być niebezpieczne.\\ 
      \textbf{Odp}:\\
      Rozwiązanie: \emph{zad1.rb}\\
      Dane: \emph{zad1\_dane.txt}\\
   \item Użyj komputera do wyznaczania wartości numerycznych kolejnych elementów ciągu ($x_{n}$).Zdefiniowanego rekurencyjnie w następujący sposób:\newline \newline
   $x_{0} = 1$\\
   $x_{1} = \frac{1}{5}$\\
   $x_{n + 2} =\frac{26}{5}x_{n+1}-x_n$\newline \newline Skomentuj otrzymane wynikii. Czy są one wiarygodne?\\ 
   Rozwiązanie:\\
   Zastanówmy się jaki jest wzór jawny tego ciągu.\\
   Niech \\ \\ $x_{n+2} = r^{n+2}$ \\ \\ Wtedy:\\ \newline
   \begin{center}
   $ r^{n+2} = \frac{26}{5}r^{n+1} - r^{n}  // :r^{n}$\newline\newline
   $r^{2} = \frac{26}{5}r^{1} - 1 $\newline\newline
   $r^{2} -\frac{26}{5}r^{1} + 1 = 0 $\newline\newline
   \end{center}
   Skorzystamy ze wzorów Viete'a:\\ \\
   \begin{center}
   $r_{1}+r_{2}=\frac{26}{5}$\\
   $r_{1}\times r_{2}=1$\\  
   \end{center}
    Więc $r_{1} = \frac{1}{5}$ a $r_{2} = 5$.\\Przedstawmy naszą zależność rekurencyjną dla 2 i 3 elementu naszego ciągu w postaci równania:\\ \\
    $
    x_{2} = \frac{1}{25} = A \times (\frac{1}{5})^{2} + B \times 5^{2} \\ \\
    x_{3} = \frac{1}{125} = A \times (\frac{1}{5})^{3} + B \times 5^{3}\\
    $
    Wnioskujemy z tego, że:\\ \\
    $
    A = 1 \\ \\
    B = 0 \\ \\
    $
    Mamy więc doczynienia z ciągiem o postaci:\\ \\
    $
    x_{n} =(\frac{1}{5})^{n}
    $
    Można zauważyć, że ciąg ten jest malejący. Przejdźmy do wyznaczenia wartości tego ciągu przy użyciu komputera.\\
    Plik:\emph{zad2.rb}\\
   Po wypisaniu 30 pierwszych elementów tego ciągu mogę stwierdzić, że otrzymane przeze mnie wynikii nie są wiarygodne, ponieważ przekazywane                     na wyjście elementy nie są w porządku malejącym, a wręcz w pewnym momencie zaczynają rosnąć, a wykazałem, że ciąg powinien byc malejący.
   \item Wykorzystując własności szeregów naprzemiennych, ustal ilu teoretycznie wyrazów szeregu: 
   \begin{center}
  \[\pi = 4\sum_{k=0}^\infty\frac{(-1)^{k}}{2k+1}\]
   \end{center}
   Należy użyć do obliczenia wartości $\pi$ z błędem mniejszym niż $10^{-7}$\\
   Z kryterium \textbf{Leibniza}\\
   Jeśli ciąg $a_{n}$ jest malejący i zbieżny do zera to szereg $\sum_{n=1}^\infty(-1)^{n}a_{n}$ jest malejący.Nasz ciąg $a_{n}$ jest postaci:
  \[a_{k}=\frac{1}{2k+1}\]
   \begin{enumerate}
   \item Czy ciąg jest zbieżny
   \[\lim_{k \rightarrow \infty}\frac{1}{2k+1} = 0\]
   Ciąg jest więc zbieżny.
   \item Czy ciąg jest malejący?
   \[\frac{1}{2k+1} > \frac{1}{2k+2}\]
   Ciąg jest malejący.
   \end{enumerate}
   Skoro wiem, że szereg jest zbieżny skorzystam z własności:
   \[|S_ - S_{k}| <= a_{k+1}\]
   Cemu tak jest?\\
   Weźmy sobie szereg naprzemienny o wyrazach kolejno równych:\\ \textbf{a1 , a2, a3, a4, a5, a6, a7, a8, a9, ...... } \\
   Ustalmy sobie S4 = a1 + a2 +a3 +a4, S - suma całości. Wiemy, że suma od a5 do an jest albo większa albo mniejsza od zera, ponieważ wyrazy możemy pogrupować w pary (a5,a6);(a7,a8) itd. Każda z tych par w zależności od wartości a5 jest albo ujemna, albo dodatnia ciąg \[a_{k}=\frac{1}{2k+1}\] jest malejący zaczynamy od liczby ujemnej więc potem dodajemy do niej dodatnią ale w module od niej mniejszą itd. Każda z tych par jest ujemna. Możemy teraz te wyrazy pogrupować tak że zostawimy sobie a5 i resztę weźmiemy w dwójki, które w zależności od porzednich np. jak były ujemne staną się dodatnie. Wiemy natomiast, że całość ma być ujemna, więc jak do ujemnej dodamy sumę tych par ( $>$ 0 ), to otrzymamy liczbę ujemną. Z tego wynika, że:\[|S_ - S_{k}| <= a_{k+1}\]
   Zależy nam na tym, aby błąd był mniejszy od $10^{-7}$ więc:
   \[|S - S_{k}| <= a_{k+1} < 10^{-7}\]
   Dalej:\\
   \begin{center}
   $
   \frac{1}{2(k+1)+1} <10^{-7} //\times(2k+1)  \newline
   1 <10^{-7}\times(2k+3) \newline
   k = 5000000 - 3 \newline$ 
   $k = 4999997 + 1 //+1 bo ma byc większe od 1 a nie równe$ \newline
   $k = 4999998  \newline\newline
   $ \end{center}
    Teraz wykonałem odpowiedni eksperyment z wykorzystaniem komputera.\\
    Źródło: \emph{zad3.rb , testzad3.rb} \\
    Wynika z niego, że dla k = 4999998 program nie wyznacza $\pi$ z błędem mniejszym niż $ 10^{-7} $. Bład jest mniejszy od $ 10^{-7} $ dla k = 5000000.
  \item Wykorzystując własności szeregów naprzemiennych, sprawdź,że do obliczenia $ \ln2 $ z błedem mniejszym niż
  $\frac{1}{2}\times 10^{-6} $ trzeba użyc ok. 2 milionów wyrazów szeregu:\\
  \[\ln x = \sum_{k=1}^\infty(-1)^{k-1}\frac{(x-1)^k}{k}\]
   Z kryterium \textbf{Leibniza}\\
   Jeśli ciąg $a_{n}$ jest malejący i zbieżny do zera to szereg $\sum_{n=1}^\infty(-1)^{n}a_{n}$ jest malejący.Nasz ciąg $a_{n}$ i dla x = 2 jest postaci:
   \[a_{k} = \frac{1}{k}\]
    \begin{enumerate}
   \item Czy ciąg jest zbieżny
   \[\lim_{k \rightarrow \infty}\frac{1}{k} = 0\]
   Ciąg jest więc zbieżny.
   \item Czy ciąg jest malejący?
   \[\frac{1}{k} < \frac{1}{k+1}\]
   Ciąg jest malejący.
   \end{enumerate}
     Skoro wiem, że szereg jest zbieżny skorzystam z własności:
   \[|S - S_{k}| <= a_{k+1}\]
    Zależy nam na tym, aby błąd był mniejszy od $\frac{1}{2}\times 10^{-6}$ więc:
   \[|S - S_{k}| <= a_{k+1} < \frac{1}{2}\times 10^{-6}\]
   Dalej \\ \\ 
   $
   \frac{1}{k+1} <  \frac{1}{2}\times 10^{-6} // \times 2 \newline \newline
   \frac{2}{k+1} <  10^{-6} \\ \\
   k + 1= 2 \times 10^{6} \\ \\
   k = 2 \times 10^{6} - 1 \\ \\
   $
   Dalej wykaż, że wykorzystanie prostego związku \[\ln 2 = \ln[e\times\frac{2}{e}]\] może znacznie przyśpieszyć obliczenia.
   \\Odp:
   \\
   \[\ln[e\times\frac{2}{e}] = \ln e + \ln \frac{2}{e} = 1 + \ln \frac{2}{e}\]
   Sprawdźmy teraz ilu elementów ciągu musimy użyć aby obliczyć $ \ln \frac{2}{e} $ z błędem mniejszym niż $\frac{1}{2}\times 10^{-6} $.
   Podstawmy do naszego szeregu za x = $ \frac{2}{e}  $. Otrzymujemy:\\
   \[\ln x = \sum_{k=1}^\infty(-1)^{k-1}\frac{(\frac{2}{e}-1)^k}{k}\]
   Szereg ten nie jest naprzemienny, co trochę komplikuje całą sprawę.\\ Zaokrąglijmy więc licznik, który jest równy ok.$ -0.264241 $ do wartości równej $ \frac{-1}{5} $ .Więc\\
  \[\ln x = \sum_{k=1}^\infty(-1)^{k-1}\frac{(\frac{-1}{5}^{k})}{k}\]
  Dalej
  Można zauważyć, że są to wartości asymptonicznie szybko malejące. Ponadto wszystkie < 0. I już różnica $ S_{11} - S_{10} $ równa jest w module $ |3.9866828560608525e-08 |$. Wiedząc, że wartości będą coraz to mniejsze, a już mamy doczynienie z liczbą praktycznie niezauważalna". Twierdzę, że do  optymalnego obliczenia ( z dość małym błędem ) sumy tego szeregu będziemy potrzebować nie więcej niż 30 jego wyrazów. ( Dalsze iteracje będą zbyt mało znaczące).
  
  
 \item Wykorzystując pomysł z poprzedniego zadania zaproponuj szybki algorytm do wyznaczania $ \ln $ bardzo dużych liczb.\\
 \begin{lstlisting}[language=Ruby, caption=Algorytm do wyznacznia logarytmu naturalnego]
def calc(x,k)
    sum = 0 
    for i in 1..k 
        sum+=(((-1) ** (i-1) ) * (((x-1) ** i) / i))
    end
    sum
end

def ln x
    if  x > 5
      temp = x / Math::E 
     return 1 + 1 + ln(temp/Math::E)
    else
     return 1 + calc((x / Math::E),30)
    end 
end

\end{lstlisting}
Opis:\\

Na początku sprawdzam czy \textbf{x} jest większy od 5, aby podczas dzielenia przez \textbf{e} wynik nie był większy od 2. Następnie wyciągam z tej liczby \textbf{e} i  do wyniku dodaje \textbf{1} i rekurencyjnie wywołuję funkcję \textbf{ln} z tym co mi zostało po wyciągnięciu e. Jeżeli argument funkcji \textbf{ln} jest mniejszy równy 5 to postępuje zgodnie z pomysłem z poprzedniego zadania.\\ \\
Test: \emph{testzad5.rb}\\

\item W języku \textbf{PWO ++} funkcja \textbf{ATAN} oblicza $ \arctan$z bardzą dużą dokłądnością, ale tylko wtedy gdy $ |x| \leq 1 $. Zaproponuj algorytm wyznaczający w jęzuku  \textbf{PWO ++} $ \arctan $ z dużą dokładnością dla $ |x| > 1 $.
 \begin{lstlisting}[language=Ruby, caption= Kod w Ruby]
def ATan(x)
    return Math.atan(x) if x.abs <= 1
    print "Wywolano funkcje ATan z |x| > 1"
end

def FATan(x)
    if x < 0
        a = - (Math::PI / 2.0)
     return a - ATan(1.0/x)
    elsif x == 0
     return ATan(x)
    else
        a = (Math::PI / 2.0)
     return a - ATan(1.0/x)
    end
end

\end{lstlisting}
Korzystam z faktu, że gdy $ x >= 0 $ to:
\[\arctan(x) + \arctan(\frac{x}{2}) =\frac{\pi}{2}\]
A gdy $ x < 0 $ to:
\[\arctan(x) + \arctan(\frac{x}{2}) =-\frac{\pi}{2}\]
 \end{enumerate}
\end{document}

