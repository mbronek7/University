\documentclass[10pt,a4paper]{article}
\usepackage[polish]{babel}
\usepackage[utf8]{inputenc}
\usepackage{polski}
\usepackage{secdot}
\title{Michał Bronikowski \protect\\ Rozwiązania wybranych zadań z drugiej listy}
\begin{document}
\maketitle
\thispagestyle{empty}
\newpage
\section{\normalsize{Jeśli klient i serwer znajdują się w odległych od siebie miejscach, zauważamy, że opóżnienie sieciowe ma zasadniczy wpływ na wydajność. Jak można temu zaradzić?}}
Można spróbować podzielić pasmo wymiany pomiędzy wszystkie komputery połączone z serwerem i wybrać kilka pośrednich, które korzystają z tych samych usług serwera do rozsyłania ich dalej.
\section{\normalsize{Co to takiego trzywarstwowa architektura klient-serwer?}}
Trzywarstwowa architektura klient-serwer to archtektura, w której serwer jest podzielony na odzielne moduły do przechowywania danych,  składowania danych i do obsługi interfejsu użytkownika. Zastosowanie tej architektury umożliwia zmianę poszczególnych modułów niezależnie od siebie.
\section{\normalsize{W czym zawiera się różnica między rozproszeniem pionowym a poziomym?}}
Różnica polega na tym, że w pionowym rozproszeniu warstwy kontaktują się wzajemnie, a w poziomym pliki są pobierane z kilku serwerów. 
\section{\normalsize{Podaj nieodparty (techniczny) argument przemawiający za tym, że polityce "coś za cos" (tit-for-tat ), stosowanej w BitTorrents, wiele brakuje do optymalnej, jeśli chodzi o dzielenie plików w Internecie.}}
Inni użytkownicy, podani przez trackera mogą znajdować się na drugim końcu świata i mogą występować różnice w czasie dostarczenia nam pakietów co spowoduje spowolnienie procesu skalania wszystkich części naszego pliku w całość.
\section{\normalsize{Współczesne samochody są napakowane elektroniką. Podaj kilka przykładów systemów samochodowych działających na zasadzie sprzężeń zwrotnych.}}
Przykłady systemów samochodowych działających na zasadzie sprzężeń zwrotnych:
\begin{itemize}
\item tempomat,
\item sterowanie temperaturą,
\item kontrola trakcji,
\item wspomaganie kierownicy.
\end{itemize}
\end{document}