\documentclass[10pt,a4paper]{article}
\usepackage[polish]{babel}
\usepackage[utf8]{inputenc}
\usepackage{polski}
\usepackage{secdot}
\title{Michał Bronikowski \protect\\ Rozwiązania wybranych zadań z pierwszej listy}
\begin{document}
\maketitle
\thispagestyle{empty}
\newpage
\section{\normalsize{Jaką rolę w systemie rozproszonym odgrywa oprogramowanie warstwy pośredniej?}}
Główną rolą warstwy pośredniej jest zapewnienie przezroczystości całemu systemowi. To znaczy że
to oprogramowanie uczestniczy w wymianie informacji pomiędzy wszystkimi jednostkami w systemie rozproszonym.
Jednocześnie użytkownik nie odczuwa, decentralizacji działań (przezroczystość). Wszelkie różnice w reprezentacji danych są ukrywane.
\section{\normalsize{Wyjaśnij, co rozumiemy przez przezroczystość (rozproszenia), i podaj przykłady różnych rodzajów przezroczystości.}}
Przezroczystośc -- inaczej transparentność -- jest cechą systemu rozproszonego. Dzięki niej użytkownik ma wrażenie, że korzysta z jednego scentralizowanego ,,komputera". \\
Teraz podam przykłady różnych rodzajów przezroczystości:
\begin{itemize}
\item {dostępu,}
\item {położenia,}
\item {wędrówki,}
\item {przemieszczenia,}
\item {zwielokrotnienia,}
\item {współbieżnośći,}
\item {awarii,}
\item {trwałości.}
\end{itemize}
\section{\normalsize{Dlaczego czasami tak trudno jest ukryć w systemie rozproszonym występowanie awarii usuwanie ich skutków (rekonstrukcję)?}}
Jest to spowodowane tym, że awaria może wystąpić w trakcie jakiegoś procesu i nie da się powrócić do stanu sprzed awarii.
Na przykład weźmy system sterujący spuszczaniem wody w wielkich zaporach wodnych (np. przy elektrowniach wodnych) i wyobraźmy sobie, że awaria nastąpiła podczas spuszczania wody ze zbiornika i zablokowane zostały klapy, które blokują wodę w pozycji otwartej. Po ustąpieniu awarii nie da się usunąć jej skutków, ponieważ nie da się wrócić do stanu sprzed wystąpienia awarii i całe miasto leżące obok tej elektrowni zostało zalane.
\section{\normalsize{Dlaczego dążenie do osiągnięcia jak największego stopnia przezroczystości nie zawsze jest dobrym pomysłem?}}
Nie jest to dobry pomysł, gdy staramy się za wszelką cenę osiągnąć przezroczystośc awarii. Jeżeli użytkownik nie będzie zdawał sobie sprawy z uszkodzenia niektórych węzłów a przykładowo administrator systemu będzie zbyt pewny siebie, może dojść do sytuacji, gdy cały system ulegnie awarii, której dałoby się uniknąć, gdyby wcześniej ktoś zareagował. Również osiągnięcie pełnej przezroczystości jest bardzo zasobożerne.
\section{\normalsize{Co to jest otwarty system rozproszony i jakie korzyści wynikają z otwartości?}}
Otwarty system rozproszony to inaczej system elastyczny, czyli taki, który można łatwo konfigurować i rozbudowywać.
Główną korzyścią tego rodzaju systemów jest redukcja kosztów związana z wymianą poszczególnych podzespołów, ponieważ wymiana jednego węzła nie pociąga za sobą wymiany kolejnych, czyli możemy unowocześniać system etapowo.
\section{\normalsize{Opisz dokładnie, co rozumiemy przez system skalowalny.}}
Skalowalność systemu rozproszonego oznacza możliwość dodawania nowych zasobów i użytkowników, rozmieszczenia poszczególnych węzłów w różnych miejscach (np. wyszukiwarka Google na całym świecie), pomimo rozrzucenia i elastyczności węzłów ich administracja jest skuteczna i szybka.
\section{\normalsize{Skalowalność można osiągnąć różnymi sposobami. Jakie to są sposoby?}} 
Te sposoby to:
\begin{itemize}
\item{zwiększenie odporności na awarie,}
\item{przerzucenie części obliczeń na komputer klienta,}
\item{spójność danych,}
\item{komunikacja asynchroniczna -- zmniejsza obciążenie sieci.}
\end{itemize}
\section{\normalsize{Wieloprocesor z 256 jednostkami centralnymi jest zorganizowany w układzie kraty o wymiarach 16 na 16. Ile wynosi najgorsze opóźnienie (w przeskokach), na jakie jest narażony komunikat}}
Najgorsze opóźnienie w tym przypadku to: 31 przeskoków.
\section{\normalsize{Na czym polega różnica między rozproszonym systemem operacyjnym a sieciowym systemem operacyjnym?}}
W sieciowym systemie operacyjnym nie ma przezroczystości, na przykład odczuwamy, że pliki nie są bezpośrednio na naszym komputerze.
\section{\normalsize{Powiedzieliśmy, że w przypadku zaniechania transakcji świat powraca do poprzedniego stanu, tak jak gdyby transakcja nigdy nie wystąpiła. Kłamaliśmy. Podaj przykład, w którym odtworzenie poprzedniego stanu świata jest niemożliwe.}}
Przykład z elektrownią z zadania 3, albo podany na ćwiczeniach przykład z drukarką, gdy awaria nastąpi w trakcie drukowania strony i nie da się wrócić do stanu sprzed awarii, ponieważ część kartki jest już zadrukowana.
\section{\normalsize{Omówiliśmy kilka przykładów wszechobecnych systemów rozproszonych. Były to systemy domowe, doglądania zdrowia indywidualnego pacjenta i sieci sensorowe. Rozszerz tę listę o więcej przykładów.}}
Dodatkowe przykłady systemów wszechobecych:
\begin{itemize}
\item{Systemy badające natężenie ruchu na autostradach,}
\item{Systemy mnierzące zanieczyszczenie powietrza w mieście,}
\item{Systemy służące do pomiarów meteorologicznych,}
\item{Sygnalizacja świetlna.}
\end{itemize}
\end{document}